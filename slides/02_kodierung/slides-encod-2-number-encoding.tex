
%<<setup-child, include = FALSE>>=
%library(knitr)
%set_parent("../style/preamble.Rnw")
%@

\input{../../2021/style/preamble4tex}
% dependencies: amsmath, amssymb, dsfont
% math spaces
\ifdefined\N
\renewcommand{\N}{\mathds{N}} % N, naturals
\else \newcommand{\N}{\mathds{N}} \fi
\newcommand{\Z}{\mathds{Z}} % Z, integers
\newcommand{\Q}{\mathds{Q}} % Q, rationals
\newcommand{\R}{\mathds{R}} % R, reals
\ifdefined\C
\renewcommand{\C}{\mathds{C}} % C, complex
\else \newcommand{\C}{\mathds{C}} \fi
\newcommand{\continuous}{\mathcal{C}} % C, space of continuous functions
\newcommand{\M}{\mathcal{M}} % machine numbers
\newcommand{\epsm}{\epsilon_m} % maximum error

% counting / finite sets
\newcommand{\setzo}{\{0, 1\}} % set 0, 1
\newcommand{\setmp}{\{-1, +1\}} % set -1, 1
\newcommand{\unitint}{[0, 1]} % unit interval

% basic math stuff
\newcommand{\xt}{\tilde x} % x tilde
\newcommand{\argmin}{\mathop{\mathrm{arg\,min}}} % argmin
\newcommand{\argmax}{\mathop{\mathrm{arg\,max}}} % argmax
\newcommand{\argminlim}{\argmin\limits} % argmin with limits
\newcommand{\argmaxlim}{\argmax\limits} % argmax with limits
\newcommand{\sign}{\operatorname{sign}} % sign, signum
\newcommand{\I}{\mathbb{I}} % I, indicator
\newcommand{\order}{\mathcal{O}} % O, order
\newcommand{\bigO}{\mathcal{O}} % Big-O Landau
\newcommand{\littleo}{{o}} % Little-o Landau
\newcommand{\pd}[2]{\frac{\partial{#1}}{\partial #2}} % partial derivative
\newcommand{\floorlr}[1]{\left\lfloor #1 \right\rfloor} % floor
\newcommand{\ceillr}[1]{\left\lceil #1 \right\rceil} % ceiling
\newcommand{\indep}{\perp \!\!\! \perp} % independence symbol

% sums and products
\newcommand{\sumin}{\sum\limits_{i=1}^n} % summation from i=1 to n
\newcommand{\sumim}{\sum\limits_{i=1}^m} % summation from i=1 to m
\newcommand{\sumjn}{\sum\limits_{j=1}^n} % summation from j=1 to p
\newcommand{\sumjp}{\sum\limits_{j=1}^p} % summation from j=1 to p
\newcommand{\sumik}{\sum\limits_{i=1}^k} % summation from i=1 to k
\newcommand{\sumkg}{\sum\limits_{k=1}^g} % summation from k=1 to g
\newcommand{\sumjg}{\sum\limits_{j=1}^g} % summation from j=1 to g
\newcommand{\summM}{\sum\limits_{m=1}^M} % summation from m=1 to M
\newcommand{\meanin}{\frac{1}{n} \sum\limits_{i=1}^n} % mean from i=1 to n
\newcommand{\meanim}{\frac{1}{m} \sum\limits_{i=1}^m} % mean from i=1 to n
\newcommand{\meankg}{\frac{1}{g} \sum\limits_{k=1}^g} % mean from k=1 to g
\newcommand{\meanmM}{\frac{1}{M} \sum\limits_{m=1}^M} % mean from m=1 to M
\newcommand{\prodin}{\prod\limits_{i=1}^n} % product from i=1 to n
\newcommand{\prodkg}{\prod\limits_{k=1}^g} % product from k=1 to g
\newcommand{\prodjp}{\prod\limits_{j=1}^p} % product from j=1 to p

% linear algebra
\newcommand{\one}{\bm{1}} % 1, unitvector
\newcommand{\zero}{\mathbf{0}} % 0-vector
\newcommand{\id}{\bm{I}} % I, identity
\newcommand{\diag}{\operatorname{diag}} % diag, diagonal
\newcommand{\trace}{\operatorname{tr}} % tr, trace
\newcommand{\spn}{\operatorname{span}} % span
\newcommand{\scp}[2]{\left\langle #1, #2 \right\rangle} % <.,.>, scalarproduct
\newcommand{\mat}[1]{\begin{pmatrix} #1 \end{pmatrix}} % short pmatrix command
\newcommand{\Amat}{\mathbf{A}} % matrix A
\newcommand{\Deltab}{\mathbf{\Delta}} % error term for vectors

% basic probability + stats
\renewcommand{\P}{\mathds{P}} % P, probability
\newcommand{\E}{\mathds{E}} % E, expectation
\newcommand{\var}{\mathsf{Var}} % Var, variance
\newcommand{\cov}{\mathsf{Cov}} % Cov, covariance
\newcommand{\corr}{\mathsf{Corr}} % Corr, correlation
\newcommand{\normal}{\mathcal{N}} % N of the normal distribution
\newcommand{\iid}{\overset{i.i.d}{\sim}} % dist with i.i.d superscript
\newcommand{\distas}[1]{\overset{#1}{\sim}} % ... is distributed as ...


\begin{document}

\lecturechapter{2}{Number encoding}
\lecture{Computerintensive Methods}


\begin{vbframe}{Codes for numbers}
\begin{itemize}
% \item Clearly, it is possible to represent every number with arbitrary preciseness in ASCII, however this also requires varying numbers of bytes for each number

%<<include = FALSE>>=
%int2bits = function(x) {
%  x = as.integer(intToBits(x))
%  paste(rev(x), collapse = "")
%}
%int2bits(23)
%int2bits(127106)
%@

% \item It is possible to fix the varying numbers of bytes, however this representation is extremely inefficient
\item The basic arithmetic operations (and many other arithmetic operations)
  are performed directly by the CPU. The fewer bits per number,
  the faster.
% \item The fewer different possibilities there are to calculate numbers
  %, the simpler the CPU becomes (or \ can be used with given
  % Complexity with available number formats more complicated things
  %).
\item For technical reasons, a number should be encoded by a
  \textbf{fixed number of bytes}, thus using $N$ bits only.
\item We are looking for a function that maps sets of numbers like $\Z$ or $\R$
   to the set of the $2^N$ available machine numbers.
% \item Many people keep dealing with computer arithmetic
  % for a more \quota{esoteric} subject.
\item A fallacy: \enquote{Computer calculations are always correct.}
% \item Unfortunately, that's not true, in many cases computers are capable of
  % not even the basic accounting types.
\item Basic knowledge of computer arithmetic is essential for anyone who mainly uses computers for calculations, i.e. especially for statisticians.
\end{itemize}

\framebreak

\enquote{Bug}-Report in \pkg{R}:
\small
\begin{verbatim}
   From: focus17@libero.it
   To: R-bugs@biostat.ku.dk
   Subject: error in trunc function
   Date: Fri,  6 Jul 2007 15:03:58 +0200 (CEST)

   the command get a wrong result

   > trunc(2.3 * 100)
   [1] 229
\end{verbatim}

Answer Duncan Murdoch:
\begin{verbatim}
   That is the correct answer. 2.3 is not
   representable exactly; the actual value used
   is slightly less.
\end{verbatim}

\framebreak

\small
\begin{verbatim}
   From: wchen@stat.tamu.edu
   To: R-bugs@biostat.ku.dk
   Subject: [Rd] match() (PR#13135)
   Date: Tue, 7 Oct 00:05:06 2008

   The match function does not return value properly.
   See an example below.

   > a = seq(0.6, 1, by = 0.01)
   > match(0.88, a)
   [1] 29
   > match(0.89, a)
   [1] NA
   ...
   > match(0.94, a)
   [1] 35
\end{verbatim}

\framebreak 

Answer Brian Ripley:
\begin{verbatim}
   FAQ Q7.31 strikes again!

   0.89 is not a member of seq(0.6,1,by=0.01), since 0.01
   cannot be represented exactly in a binary computer.
\end{verbatim}


\framebreak

\small
\begin{verbatim}
From: Friedrich Leisch <friedrich.leisch@stat.uni-muenchen.de>
To: Antonio Linan <antoniolvsa@hotmail.com>
Cc: <cran@r-project.org>
Subject: Re: Bug in R?
Date: Thu, 5 Nov 2009 13:57:03 +0100

>>>>> On Thu, 5 Nov 2009 13:35:09 +0100,
>>>>> Antonio Linan (AL) wrote:

  > Hi, I'm not sure if it's really a bug:
  > When you execute:
  >> (2 / 3) * (0.6 / (1 - 0.6))
  > the result will be:
  > [1] 1
  > but if you execute:
  >> (2 / 3) * (0.6 / (1 - 0.6)) == 1
  > the result is:
  > [1] FALSE
  > Note: I'm using version 2.9.2, (and tried it in
  >   2.9.1 in 2.9.1 too) with Microsoft Windows XP
  >   [Version 5.1.2600].
  > Thank you.

FAQ 7.31 strikes again:
R> 1 - (2 / 3) * (0.6 / (1 - 0.6))
[1] 2.220446e-16
R> .Machine$double.eps
[1] 2.220446e-16
\end{verbatim}

\framebreak
\vspace*{-0.5cm}
\begin{verbatim}
From: Marc Schwartz <marc_schwartz_at_me.com>
Date: Fri, 09 Jul 2010 09:00:10 -0500

On Jul 9, 2010, at 8:46 AM, Trafim Vanishek wrote:
  > Dear all,
  >
  > might seem an easy question but I cannot figure it out.
  >
  > floor(100 * (.58))
  > [1] 57
  >
  > where is the trick here?
  >   And how can I end up with the right answer?

See R FAQ 7.31
> sprintf("%.20f", 100 * .58)
[1] "57.99999999999999289457"
\end{verbatim}
\end{vbframe}


\endlecture
\end{document}

% https://github.com/SurajGupta/r-source/blob/56becd21c75d104bfec829f9c23baa2e144869a2/src/library/stats/src/cov.c

