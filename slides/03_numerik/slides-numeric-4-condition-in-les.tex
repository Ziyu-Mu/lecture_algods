
%<<setup-child, include = FALSE>>=
%library(knitr)
%set_parent("../style/preamble.Rnw")
%options(digits = 16)
%@
\input{../../2021/style/preamble4tex}
% dependencies: amsmath, amssymb, dsfont
% math spaces
\ifdefined\N
\renewcommand{\N}{\mathds{N}} % N, naturals
\else \newcommand{\N}{\mathds{N}} \fi
\newcommand{\Z}{\mathds{Z}} % Z, integers
\newcommand{\Q}{\mathds{Q}} % Q, rationals
\newcommand{\R}{\mathds{R}} % R, reals
\ifdefined\C
\renewcommand{\C}{\mathds{C}} % C, complex
\else \newcommand{\C}{\mathds{C}} \fi
\newcommand{\continuous}{\mathcal{C}} % C, space of continuous functions
\newcommand{\M}{\mathcal{M}} % machine numbers
\newcommand{\epsm}{\epsilon_m} % maximum error

% counting / finite sets
\newcommand{\setzo}{\{0, 1\}} % set 0, 1
\newcommand{\setmp}{\{-1, +1\}} % set -1, 1
\newcommand{\unitint}{[0, 1]} % unit interval

% basic math stuff
\newcommand{\xt}{\tilde x} % x tilde
\newcommand{\argmin}{\mathop{\mathrm{arg\,min}}} % argmin
\newcommand{\argmax}{\mathop{\mathrm{arg\,max}}} % argmax
\newcommand{\argminlim}{\argmin\limits} % argmin with limits
\newcommand{\argmaxlim}{\argmax\limits} % argmax with limits
\newcommand{\sign}{\operatorname{sign}} % sign, signum
\newcommand{\I}{\mathbb{I}} % I, indicator
\newcommand{\order}{\mathcal{O}} % O, order
\newcommand{\bigO}{\mathcal{O}} % Big-O Landau
\newcommand{\littleo}{{o}} % Little-o Landau
\newcommand{\pd}[2]{\frac{\partial{#1}}{\partial #2}} % partial derivative
\newcommand{\floorlr}[1]{\left\lfloor #1 \right\rfloor} % floor
\newcommand{\ceillr}[1]{\left\lceil #1 \right\rceil} % ceiling
\newcommand{\indep}{\perp \!\!\! \perp} % independence symbol

% sums and products
\newcommand{\sumin}{\sum\limits_{i=1}^n} % summation from i=1 to n
\newcommand{\sumim}{\sum\limits_{i=1}^m} % summation from i=1 to m
\newcommand{\sumjn}{\sum\limits_{j=1}^n} % summation from j=1 to p
\newcommand{\sumjp}{\sum\limits_{j=1}^p} % summation from j=1 to p
\newcommand{\sumik}{\sum\limits_{i=1}^k} % summation from i=1 to k
\newcommand{\sumkg}{\sum\limits_{k=1}^g} % summation from k=1 to g
\newcommand{\sumjg}{\sum\limits_{j=1}^g} % summation from j=1 to g
\newcommand{\summM}{\sum\limits_{m=1}^M} % summation from m=1 to M
\newcommand{\meanin}{\frac{1}{n} \sum\limits_{i=1}^n} % mean from i=1 to n
\newcommand{\meanim}{\frac{1}{m} \sum\limits_{i=1}^m} % mean from i=1 to n
\newcommand{\meankg}{\frac{1}{g} \sum\limits_{k=1}^g} % mean from k=1 to g
\newcommand{\meanmM}{\frac{1}{M} \sum\limits_{m=1}^M} % mean from m=1 to M
\newcommand{\prodin}{\prod\limits_{i=1}^n} % product from i=1 to n
\newcommand{\prodkg}{\prod\limits_{k=1}^g} % product from k=1 to g
\newcommand{\prodjp}{\prod\limits_{j=1}^p} % product from j=1 to p

% linear algebra
\newcommand{\one}{\bm{1}} % 1, unitvector
\newcommand{\zero}{\mathbf{0}} % 0-vector
\newcommand{\id}{\bm{I}} % I, identity
\newcommand{\diag}{\operatorname{diag}} % diag, diagonal
\newcommand{\trace}{\operatorname{tr}} % tr, trace
\newcommand{\spn}{\operatorname{span}} % span
\newcommand{\scp}[2]{\left\langle #1, #2 \right\rangle} % <.,.>, scalarproduct
\newcommand{\mat}[1]{\begin{pmatrix} #1 \end{pmatrix}} % short pmatrix command
\newcommand{\Amat}{\mathbf{A}} % matrix A
\newcommand{\Deltab}{\mathbf{\Delta}} % error term for vectors

% basic probability + stats
\renewcommand{\P}{\mathds{P}} % P, probability
\newcommand{\E}{\mathds{E}} % E, expectation
\newcommand{\var}{\mathsf{Var}} % Var, variance
\newcommand{\cov}{\mathsf{Cov}} % Cov, covariance
\newcommand{\corr}{\mathsf{Corr}} % Corr, correlation
\newcommand{\normal}{\mathcal{N}} % N of the normal distribution
\newcommand{\iid}{\overset{i.i.d}{\sim}} % dist with i.i.d superscript
\newcommand{\distas}[1]{\overset{#1}{\sim}} % ... is distributed as ...


\begin{document}

\lecturechapter{3}{Condition in systems of linear equations (LES)}
\lecture{CIM1 Statistical Computing}

\begin{vbframe}{Condition number: matrix multiplication}
Although we are actually interested in $\xv = \Amat^{-1}\yv$, we first consider the direct problem: $\yv = \Amat \xv$.

\lz

Since
$$
y_i = \sum_{j = 1}^n A_{ij}x_j,
$$
it is to be expected that the high condition number of the addition is also transferred to $\Amat \xv$.

\framebreak

When $\Amat$ and $\xv$ are disturbed by $\Delta \Amat$ and $\Delta \xv$, the \textbf{absolute error} in the result $\yv$ is:

\begin{eqnarray*}
\yv + \Delta \yv &=& (\Amat + \Delta \Amat)(
  \xv + \Delta \xv)\\  &=& \Amat \xv + \Delta \Amat\xv + \Amat\Delta \xv +
      \Delta \Amat\Delta \xv \qquad | - \yv \\
\Delta \yv &=& \Delta \Amat\xv + \Amat\Delta \xv +
      \Delta \Amat\Delta \xv
\end{eqnarray*}

The absolute error is therefore estimated as follows
\begin{eqnarray*}
\to \quad \|\Delta \yv\| &=& \|\Delta \Amat\xv + \Amat\Delta \xv +
      \Delta \Amat\Delta \xv\| \\ &\le& \|\Delta \Amat\|\|\xv\| + \|\Amat\|\|\Delta \xv\| +
      \|\Delta \Amat\|\|\Delta \xv\| \\
      &\approx& \|\Delta \Amat\|\|\xv\| + \|\Amat\|\|\Delta \xv\|
\end{eqnarray*}

% \framebreak
%
% Kondition Addition:\\
% \medskip
% $\quad y = x_1 + x_2,\quad \tilde y = (x_1 + \delta_1) + (x_2 + \delta_2)$
% \begin{eqnarray*}
%  \SpAr \frac{\tilde y - y}{y} &=& \frac{\delta_1 + \delta_2}{x_1 + x_2} \\
%    &=& \frac{x_1}{x_1 + x_2}\ \frac{\delta_1}{x_1} +  \frac{x_2}{x_1 + x_2}\ \frac{\delta_2}{x_2}
% \end{eqnarray*}
% Schlecht konditioniert, falls $x_1 + x_2$ nahe bei 0 ($\kappa$ wird beliebig groß).\\
% \bigskip
% Da
% $$
% y_i = \sum_{j = 1}^n A_{ij}x_j,
% $$
% ist zu erwarten, dass Kondition sich auch auf $\Amat \xv$ überträgt.

\framebreak

From the estimate for the absolute error, we obtain
\small
\begin{eqnarray*}
\|\Delta \yv\| &\leq& \|\Delta \Amat\|\|\xv\| +
  \|\Amat\|\|\Delta \xv\| \\
&=& (\|\Delta \Amat\|\|\xv\| +
  \|\Amat\|\|\Delta \xv\|) \frac{\|\xv\|}{\|\xv\|} \\
&=& (\|\Delta \Amat\|\|\xv\| +
  \|\Amat\|\|\Delta \xv\|) \frac{\|\Amat^{-1}\yv\|}{\|\xv\|} \\
&\leq& (\|\Delta \Amat\|\|\xv\| +
  \|\Amat\|\|\Delta \xv\|) \frac{\|\Amat^{-1}\|\|\yv\|}{\|\xv\|} \\
&=& \left(\frac{\|\Amat\|}{\|\Amat\|}\|\Delta \Amat\|\|\xv\| +
  \|\Amat\|\|\Delta \xv\|\right) \frac{\|\Amat^{-1}\|\|\yv\|}{\|\xv\|} \\
&=& \left(\frac{\|\Amat\|\|\Amat^{-1}\|}{\|\Amat\|\|\xv\|}\|\Delta \Amat\|\|\xv\| +
  \|\Amat\|\|\Amat^{-1}\|\frac{\|\Delta \xv\|}{\|\xv\|}\right) \|\yv\| \\
\frac{\|\Delta \yv\|}{\|\yv\|} &\leq& \|\Amat\|\|\Amat^{-1}\|
  \left(\frac{\|\Delta \Amat\|}{\|\Amat\|} + \frac{\|\Delta \xv\|}{\|\xv\|} \right).
\end{eqnarray*}

\framebreak

The condition number
$$
\kappa(\Amat) = \|\Amat\|\|\Amat^{-1}\|
$$
describes the propagation of relative errors both in the matrix and on the right side of the LES.\\
\medskip
Example:
$$
\yv = \Amat \xv, \quad \Amat =
\mat{1 & 1 \\ 1 & 1 + \epsilon}, \quad \xv^T = \mat{ 1 & -1 }
$$
$$
\|\Amat\|_{\infty} = 2 + \epsilon, \quad \Amat^{-1} = \frac{1}{1 + \epsilon - 1}
\mat{ 1 + \epsilon & -1 \\ -1 & 1}, \quad \|\Amat^{-1}\|_\infty = \frac{2 + \epsilon}{\epsilon}
$$
$$
\kappa(\Amat) = (2 + \epsilon)\frac{2 + \epsilon}{\epsilon} =
  \frac{4 + 4\epsilon + \epsilon^2}{\epsilon} \approx \frac{4}{\epsilon} \quad
  \text{ and}\quad \epsilon \ll 1.
$$
$\epsilon = 10^{-8}$ and machine epsilon $\texttt{eps} = 10^{-16}$ in $\xv$ result in
a relative precision of just about $4 / 10^{-8} \cdot 10^{-16} = 4 \cdot 10^{-8}$ in
$\yv$.
\end{vbframe}



\begin{vbframe}{Condition number: LES}
% Wenn
% $$
% \Amat \xv = \yv, \quad \Amat(\xv + \Delta \xv) =
%   y + \Delta \yv,
% $$
% gilt für den relativen Fehler die Abschätzung
% $$
% \frac{\|\Delta \xv\|}{\|\xv\|} =
%   \kappa(\Amat) \frac{\|\Delta \yv\|}{\|\yv\|}.
% $$

The solution of the LES $\xv = \Amat^{-1}\yv$ is (numerically) equivalent to the results of the matrix multiplication. The following applies

$$
\kappa(\Amat) = \|\Amat\|\|\Amat^{-1}\| = \kappa(\Amat^{-1}).
$$

\textbf{Note:}
$$
\kappa(\Amat) = \|\Amat\|\|\Amat^{-1}\| \geq \|\Amat\Amat^{-1}\| = \| \mathbf{I} \| = 1.
$$

%\framebreak
\lz
\textbf{Example 1:}

Consider the matrix $\Amat = \mat{1 & 1 \\1 + 10^{-10} & 1 - 10^{-10}}$
\lz
% <<>>=
% A = matrix(c(1, 1 + 10e-10, 1, 1 - 10e-10), nrow = 2)
% @

The matrix is ill-conditioned with condition number $\kappa \approx 2 \times 10^9$.

For the solution of the LES $\Amat \xv = \yv$ this means: a small variation in the input data (e.g. $\yv=(1, 1) \to \tilde \yv = (1, 1.00001)$) leads to a big change in the solution.
\lz
\lz
\footnotesize
\begin{verbbox}
A = matrix(c(1, 1 + 10e-10, 1, 1 - 10e-10), nrow = 2)
y = c(1, 1)
yt = c(1, 1.00001)
\end{verbbox}
\col
\vspace{0.1cm}

\begin{verbbox}
solve(A, y)
## [1] 0.4999999722444252 0.5000000277555748
\end{verbbox}
\col
\vspace{0.1cm}

\begin{verbbox}
solve(A, yt)
## [1] 5000.499858862901 -4999.499858862901
\end{verbbox}
\col

\normalsize

\framebreak

\textbf{Example 2:} The Hilbert matrix is known to be ill-conditioned!
$$
H_{ij} = \frac{1}{i + j - 1},
$$
\footnotesize
\begin{verbbox}
hilbert = function(n) {
  i = 1:n 
  return(1 / outer(i - 1, i, "+"))
}
hilbert(4)
\end{verbbox}
\col
\vspace{0.2cm}
\begin{verbbox}
##                    [,1]               [,2]
## [1,] 1.0000000000000000 0.5000000000000000
## [2,] 0.5000000000000000 0.3333333333333333
## [3,] 0.3333333333333333 0.2500000000000000
## [4,] 0.2500000000000000 0.2000000000000000
##                    [,3]               [,4]
## [1,] 0.3333333333333333 0.2500000000000000
## [2,] 0.2500000000000000 0.2000000000000000
## [3,] 0.2000000000000000 0.1666666666666667
## [4,] 0.1666666666666667 0.1428571428571428
\end{verbbox}
\col
\framebreak

\footnotesize
\begin{verbbox}
foo = function(n) {
  cond = sapply(n, function(i) {
    norm(hilbert(i)) * norm(solve(hilbert(i)))
  })
  return(cbind(n, cond))
}
foo(4:10)                                                                                                                                                                                                                
##      n                  cond
## [1,] 4 2.837499999999738e+04
## [2,] 5 9.436559999999363e+05
## [3,] 6 2.907027900294877e+07
## [4,] 7 9.851948897194694e+08
## [5,] 8 3.387279082022739e+10
## [6,] 9 1.099650993366049e+12
## [7,] 10 3.535372424347476e+13
\end{verbbox}
\col
\end{vbframe}


\begin{vbframe}{well- vs. ill-posed problems}
A problem is called well-posed if the following holds:
\begin{itemize}
\item There exists a solution for the problem
\item The existing solution is unique
\item The solution depends continuously on the condition of the problem (stable)
\end{itemize}
A problem is called ill-posed if it violates at least one of these properties. However, the instability of solutions usually causes the most difficulties.
\end{vbframe}


\begin{vbframe}{\enquote{Do not invert that matrix}}

\textbf{Important:} \textbf{Never} solve an LES (numerically) using $\xv = \Amat^{-1} \yv$.
\lz
\footnotesize
\begin{verbbox}
solve(A) %*% y
\end{verbbox}
\col
\lz

\normalsize
Although theoretically correct, internally \textbf{two} (possibly ill-posed) problems are solved:

\begin{itemize}
\item Inversion of $\Amat$ (solution of $\Amat \xv = \bm{0}$) has a condition of $\|\Amat\|\|\Amat^{-1}\|$
\item The multiplication of $\Amat^{-1} \cdot \yv$ has a condition of $\|\Amat\|\|\Amat^{-1}\|$
\end{itemize}

The condition inflates: $\|\Amat\|^2\|\Amat^{-1}\|^2$

\framebreak

\textbf{Better:} Solve directly by
\lz
\footnotesize
\begin{verbbox}
solve(A, y)
\end{verbbox}
\col
\lz
\normalsize
\textbf{Advantages:}

\begin{itemize}
\item \textbf{Stability}: In the worst case only \textbf{one} ill-posed subproblem is solved.
\item \textbf{Memory}: The $n^2$ entries of the inverted matrix $\bm{A}^{-1}$ must be saved. With a direct solution via the LES only $\bm{x}\in \R^n$ is stored ($\frac{1}{n}$-th of storage space).
% \item \textbf{Laufzeit}: Lösungsalgorithmen für die direkte Lösung von $Ax = b$ sind schneller
\end{itemize}

Systems of equations can be solved efficiently and numerically stable by means of matrix decompositions (more on this in chapter 7 - matrix decompositions).

\end{vbframe}


% \begin{vbframe}{Dos and Don'ts}
% \end{vbframe}

\begin{vbframe}{Sherman-Morrison formula}

If a matrix $\bm{X}$ can be represented by $\bm{X} = \Amat + \bm{u}\bm{v}^T$, $\bm{X}^{-1}$ can be calculated using the \textbf{Sherman-Morrison formula} as follows:

$$\bm{X}^{-1} = (\Amat + \bm{u}\bm{v}^T)^{-1} = \Amat^{-1} -  \frac{\Amat^{-1}\bm{u}\bm{v}^T\Amat^{-1}}{1-\bm{v}^T\Amat^{-1}\bm{u}}$$

\textbf{Proof:}
\vspace*{-.5cm}
\begin{eqnarray*}
\bm{X} \cdot \bm{X}^{-1} &=& (\Amat + \bm{u}\bm{v}^T) (\Amat^{-1} -  \frac{\Amat^{-1}\bm{u}\bm{v}^T\Amat^{-1}}{1-\bm{v}^T\Amat^{-1}\bm{u}}) \\ &=& \Amat\Amat^{-1} + \bm{u}\bm{v}^T \Amat^{-1} - \frac{\Amat\Amat^{-1}\bm{u}\bm{v}^T\Amat^{-1} - \bm{u}\bm{v}^T\Amat^{-1}\bm{u}\bm{v}^T\Amat^{-1}}{1-\bm{v}^T\Amat^{-1}\bm{u}}   \\
&=& \id + \bm{u}\bm{v}^T \Amat^{-1} - \frac{\bm{u}\bm{v}^T\Amat^{-1} - \bm{u}\bm{v}^T\Amat^{-1}\bm{u}\bm{v}^T\Amat^{-1}}{1-\bm{v}^T\Amat^{-1}\bm{u}}  \\ &=&  \id + \bm{u}\bm{v}^T \Amat^{-1} - \frac{\bm{u} (1 - \bm{v}^T\Amat^{-1}\bm{u}) \bm{v}^T\Amat^{-1}}{1-\bm{v}^T\Amat^{-1}\bm{u}}\\ &=& \id + \bm{u}\bm{v}^T\Amat^{-1} - \bm{u}\bm{v}^T\Amat^{-1} = \id
\end{eqnarray*}

\end{vbframe}

\begin{vbframe}{Woodbury formula}

If a matrix $\bm{X}$ can be represented by $\bm{X} = \Amat + \bm{U}\bm{C}\bm{V}$, $\bm{X}^{-1}$ can be calculated using the \textbf{Woodbury formula} as:

$$\bm{X}^{-1} = (\Amat + \bm{U}\bm{C}\bm{V})^{-1} = \Amat^{-1} - \Amat^{-1}\bm{U}(\bm{C}^{-1} + \bm{V}\Amat^{-1}\bm{U})^{-1}\bm{V}\Amat^{-1}$$

The formula is especially useful if $\Amat^{-1}$ is very easy to calculate or has already been calculated.

\lz

The Woodbury formula is often used in optimization (low-rank updates, BFGS updates). See Chapter 10 (Multivariate Optimization) for more information.

% Typische Beispiele:
% \begin{itemize}
% \item Berechnung der Inversen des Low-Rank-Updates von $\Amat \leftarrow \Amat + \bm{U}\bm{C}\bm{V}$
% \item Approximation der Inversen von $\Amat+\bm{B}$, wobei $\bm{B} \approx \bm{U}\bm{C}\bm{V}$ durch Singulärwertzerlegung.
% \end{itemize}

\end{vbframe}

\endlecture
\end{document}
