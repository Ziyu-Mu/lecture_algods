
%<<setup-child, include = FALSE>>=
%library(knitr)
%options(digits = 16)

%library(RCurl)
%library(XML)
%library(tm)
%library(NMF)
%library(microbenchmark)
%library(ggplot2)
%library(wordcloud)
%set_parent("../style/preamble.Rnw")
%@


\newcommand{\xdownarrow}[1]{%
  {\left\downarrow\vbox to #1{}\right.\kern-\nulldelimiterspace}
}

\newcommand{\grey}[1]{\textcolor{grey}{#1}}
\newcommand{\red}[1]{\textcolor{red}{#1}}

\input{../../2021/style/preamble4tex}
% dependencies: amsmath, amssymb, dsfont
% math spaces
\ifdefined\N
\renewcommand{\N}{\mathds{N}} % N, naturals
\else \newcommand{\N}{\mathds{N}} \fi
\newcommand{\Z}{\mathds{Z}} % Z, integers
\newcommand{\Q}{\mathds{Q}} % Q, rationals
\newcommand{\R}{\mathds{R}} % R, reals
\ifdefined\C
\renewcommand{\C}{\mathds{C}} % C, complex
\else \newcommand{\C}{\mathds{C}} \fi
\newcommand{\continuous}{\mathcal{C}} % C, space of continuous functions
\newcommand{\M}{\mathcal{M}} % machine numbers
\newcommand{\epsm}{\epsilon_m} % maximum error

% counting / finite sets
\newcommand{\setzo}{\{0, 1\}} % set 0, 1
\newcommand{\setmp}{\{-1, +1\}} % set -1, 1
\newcommand{\unitint}{[0, 1]} % unit interval

% basic math stuff
\newcommand{\xt}{\tilde x} % x tilde
\newcommand{\argmin}{\mathop{\mathrm{arg\,min}}} % argmin
\newcommand{\argmax}{\mathop{\mathrm{arg\,max}}} % argmax
\newcommand{\argminlim}{\argmin\limits} % argmin with limits
\newcommand{\argmaxlim}{\argmax\limits} % argmax with limits
\newcommand{\sign}{\operatorname{sign}} % sign, signum
\newcommand{\I}{\mathbb{I}} % I, indicator
\newcommand{\order}{\mathcal{O}} % O, order
\newcommand{\bigO}{\mathcal{O}} % Big-O Landau
\newcommand{\littleo}{{o}} % Little-o Landau
\newcommand{\pd}[2]{\frac{\partial{#1}}{\partial #2}} % partial derivative
\newcommand{\floorlr}[1]{\left\lfloor #1 \right\rfloor} % floor
\newcommand{\ceillr}[1]{\left\lceil #1 \right\rceil} % ceiling
\newcommand{\indep}{\perp \!\!\! \perp} % independence symbol

% sums and products
\newcommand{\sumin}{\sum\limits_{i=1}^n} % summation from i=1 to n
\newcommand{\sumim}{\sum\limits_{i=1}^m} % summation from i=1 to m
\newcommand{\sumjn}{\sum\limits_{j=1}^n} % summation from j=1 to p
\newcommand{\sumjp}{\sum\limits_{j=1}^p} % summation from j=1 to p
\newcommand{\sumik}{\sum\limits_{i=1}^k} % summation from i=1 to k
\newcommand{\sumkg}{\sum\limits_{k=1}^g} % summation from k=1 to g
\newcommand{\sumjg}{\sum\limits_{j=1}^g} % summation from j=1 to g
\newcommand{\summM}{\sum\limits_{m=1}^M} % summation from m=1 to M
\newcommand{\meanin}{\frac{1}{n} \sum\limits_{i=1}^n} % mean from i=1 to n
\newcommand{\meanim}{\frac{1}{m} \sum\limits_{i=1}^m} % mean from i=1 to n
\newcommand{\meankg}{\frac{1}{g} \sum\limits_{k=1}^g} % mean from k=1 to g
\newcommand{\meanmM}{\frac{1}{M} \sum\limits_{m=1}^M} % mean from m=1 to M
\newcommand{\prodin}{\prod\limits_{i=1}^n} % product from i=1 to n
\newcommand{\prodkg}{\prod\limits_{k=1}^g} % product from k=1 to g
\newcommand{\prodjp}{\prod\limits_{j=1}^p} % product from j=1 to p

% linear algebra
\newcommand{\one}{\bm{1}} % 1, unitvector
\newcommand{\zero}{\mathbf{0}} % 0-vector
\newcommand{\id}{\bm{I}} % I, identity
\newcommand{\diag}{\operatorname{diag}} % diag, diagonal
\newcommand{\trace}{\operatorname{tr}} % tr, trace
\newcommand{\spn}{\operatorname{span}} % span
\newcommand{\scp}[2]{\left\langle #1, #2 \right\rangle} % <.,.>, scalarproduct
\newcommand{\mat}[1]{\begin{pmatrix} #1 \end{pmatrix}} % short pmatrix command
\newcommand{\Amat}{\mathbf{A}} % matrix A
\newcommand{\Deltab}{\mathbf{\Delta}} % error term for vectors

% basic probability + stats
\renewcommand{\P}{\mathds{P}} % P, probability
\newcommand{\E}{\mathds{E}} % E, expectation
\newcommand{\var}{\mathsf{Var}} % Var, variance
\newcommand{\cov}{\mathsf{Cov}} % Cov, covariance
\newcommand{\corr}{\mathsf{Corr}} % Corr, correlation
\newcommand{\normal}{\mathcal{N}} % N of the normal distribution
\newcommand{\iid}{\overset{i.i.d}{\sim}} % dist with i.i.d superscript
\newcommand{\distas}[1]{\overset{#1}{\sim}} % ... is distributed as ...


\begin{document}

\lecturechapter{7}{Gaussian Elimination (LU Decomposition)}
\lecture{CIM1 Statistical Computing}

% \begin{vbframe}{Notation}
% 
% \begin{itemize}
% \item $\xv = (x_1, x_2, ..., x_n)^\top$: Vector in $\R^n$
% \item $\bm{e}_i:=(0, ..., 0, \underbrace{1}_{\text{Position $i$}}, 0, ..., 0)$: i-th canonical unit vector 
% \item $\Amat$: Matrix in $\R^{m \times n}$
% \item $\Amat \ge 0$: $a_{ij} \ge 0$ for all $i, j$ (non-negative matrix)
% \item $\|\Amat\|$: Matrix norm, e.g. 
% \begin{itemize}
% \item $\|\Amat\|_1 = \max_j\left(\sum_i |a_{ij}|  \right)$ (Column sum norm)
% \item $\|\Amat\|_2 = \left(\mbox{largest eigenvalue of } \Amat^\top\Amat  \right)^{1/2}$ (Euclidean norm)
% \item $\|\Amat\|_\infty = \max_i\left(\sum_j |a_{ij}|  \right)$ (Row sum norm)
% \item $\|\Amat\|_F = \sqrt{\sum_{i=1}^m\sum_{j=1}^n |a_{ij}|^2}$ (Frobenius norm)
% \end{itemize}
% \item $\kappa$: Condition number
% \end{itemize}
% 
% \framebreak
% 
% The following formulas are commonly used in this chapter
% 
% \lz
% 
% $$
% \sum_{k = 1}^n k = \frac{n(n+1)}{2}
% $$
% 
% \lz
% 
% (Sum of natural numbers by Carl Friedrich Gauss) as well as 
% 
% $$
% \sum_{k = 1}^n k^2 = \frac{n(n+1)(2n + 1)}{6}
% $$
% 
% \end{vbframe}

%\section{Matrix Decompositions to solve Systems of Linear Equations (LES)}



%\section{Algorithms to solve Systems of Linear Equations}

\begin{vbframe}{Gaussian elimination (LU decomposition)}
\textbf{Aim:} Solve LES of the form $\Amat \xv = \boldsymbol{b}$\\
\medskip

with $\Amat \in \R^{n \times n}$ \textbf{regular} (invertible), $\xv \in \R^n$ and $\boldsymbol{b} \in \R^n$.

\begin{enumerate}
\item Calculate $\Amat = \mathbf{LU}$ (or $\mathbf{PA} = \mathbf{LU}$), \\
  where $\mathbf{L}$ is a normalized lower triangular matrix, $\mathbf{U}$ is an upper triangular matrix, and $\mathbf{P}$ is a permutation matrix.

\begin{footnotesize}
$$
\Amat = \mat{
1       & 0     & \cdots  & 0 \\
l_{21}  & \ddots& \ddots  & \vdots \\
\vdots  & \ddots& \ddots  & 0 \\
l_{n1}  & \cdots& l_{n(n-1)}  & 1 } \cdot
\mat{
u_{11}  & \cdots  & \cdots & u_{1n} \\
0       & \ddots  & \ddots & \vdots \\
\vdots  & \ddots  & \ddots  & \vdots \\
0       & \cdots  & 0       & u_{nn} }$$
\end{footnotesize}

\item Solve  $\mathbf{L}\textcolor{blue}{\textbf{y}} ( = \mathbf{L(Ux)} = \mathbf{Ax}) = \mathbf{b}$.
\item Solve  $\mathbf{U}\textcolor{green}{\textbf{x}} = \textcolor{blue}{\textbf{y}}$.
\end{enumerate}



\framebreak

Let $\Amat \xv = \boldsymbol{b}$ be a LES
$$
\begin{pmatrix*}[r]
2 & 8 & 1 \\
4 & 4 & -1 \\
-1 & 2 & 12\end{pmatrix*}
\mat{
x_1 \\ x_2 \\ x_3
\end{pmatrix} = \begin{pmatrix}
32 \\ 16 \\ 52 }
$$

\begin{enumerate}
\item $\Amat = \mathbf{LU}$\\

To convert $\Amat$ into an upper triangular matrix, we need 3
elementary transformations of type III:
\footnotesize
\begin{eqnarray*}
\mat{
2 & 8 & 1 \\
4 & 4 & -1 \\
-1 & 2 & 12 }
\begin{array}{l}
 \\
Z_2 - 2Z_1 \\
Z_3 + \frac{1}{2} Z_1
\end{array} &\rightarrow& \mat{
2 & 8 & 1 \\
0 & -12 & -3 \\
0 & 6 & \frac{25}{2} }
\begin{array}{l}
 \\
 \\
 Z_3 + \frac{1}{2} Z_2
\end{array} \\
&\rightarrow& \mat{
2 & 8 & 1 \\
0 & -12 & -3 \\
0 & 0 & 11 }
= \mathbf{U}.
\end{eqnarray*} \normalsize

\framebreak

If we write these transformations in matrix notation, we obtain

\footnotesize
$$
\mathbf{T}_3\mathbf{T}_2\mathbf{T}_1 = \begin{pmatrix*}[r]
1 & 0 & 0 \\
0 & 1 & 0 \\
0 & \frac{1}{2} & 1\end{pmatrix*}
\begin{pmatrix*}[r]
1 & 0 & 0 \\
0 & 1 & 0 \\
\frac{1}{2} & 0 & 1 \end{pmatrix*}
\begin{pmatrix*}[r]
1 & 0 & 0 \\
-2 & 1 & 0 \\
0 & 0 & 1 \end{pmatrix*}
=
\begin{pmatrix*}[r]
1 & 0 & 0 \\
-2 & 1 & 0 \\
-\frac{1}{2} & \frac{1}{2} & 1 \end{pmatrix*}
$$ \normalsize
Hence,
\footnotesize
$$
\mathbf{T}_3\mathbf{T}_2\mathbf{T}_1\Amat =
\begin{pmatrix*}[r]
1 & 0 & 0 \\
-2 & 1 & 0 \\
-\frac{1}{2} & \frac{1}{2} & 1 \end{pmatrix*}
\begin{pmatrix*}[r]
2 & 8 & 1 \\
4 & 4 & -1 \\
-1 & 2 & 12\end{pmatrix*} =
\mat{
2 & 8 & 1 \\
0 & -12 & -3 \\
0 & 0 & 11 }
= \mathbf{U}
$$ \normalsize
and
\footnotesize
$$
\Amat = \mathbf{T}_1^{-1}\mathbf{T}_2^{-1}\mathbf{T}_3^{-1}\mathbf{U} = \mathbf{LU}
$$
\normalsize
with
\footnotesize
$$
\mathbf{L} = \begin{pmatrix*}[r]
1 & 0 & 0 \\
2 & 1 & 0 \\
0 & 0 & 1\end{pmatrix*}
\begin{pmatrix*}[r]
1 & 0 & 0 \\
0 & 1 & 0 \\
-\frac{1}{2} & 0 & 1 \end{pmatrix*}
\begin{pmatrix*}[r]
1 & 0 & 0 \\
0 & 1 & 0 \\
0 & -\frac{1}{2} & 1 \end{pmatrix*} =
\mat{
1 & 0 & 0 \\
2 & 1 & 0 \\
-\frac{1}{2} & -\frac{1}{2} & 1 }.
$$
\normalsize
\framebreak

\textbf{General Theory}:\\
\medskip

The so-called \emph{Frobenius matrix}

$$
\mathbf{T}_k = \mathbf{I} - \mathbf{c}_k\mathbf{e}_k^\top
$$
with \scriptsize
$$
\mathbf{c}_k = \mat{
0 \\
0 \\
\vdots \\
0 \\
\mu_{k + 1} \\
\vdots \\
\mu_n },
\quad \mathbf{e}_k = \mat{
0 \\
\vdots \\
0 \\
1 \\
0 \\
\vdots \\
0 }, \quad
\mathbf{T}_k = \mat{
1      & 0      & \cdots & 0            & 0      & \cdots & 0 \\
0      & 1      & \cdots & 0            & 0      & \cdots & 0 \\
\vdots & \vdots & \ddots & \vdots       & \vdots &        & \vdots \\
0      & 0      & \cdots & 1            & 0      & \cdots & 0 \\
0      & 0      & \cdots & -\mu_{k + 1} & 1      & \cdots & 0 \\
\vdots & \vdots &        & \vdots       & \vdots & \ddots & \vdots \\
0      & 0      & \cdots & -\mu_n       & 0      & \cdots & 1 },
$$ \normalsize
and $\mathbf{T}_k^{-1} = \mathbf{I} + \mathbf{c}_k\mathbf{e}_k^\top$.

\framebreak

Any type III row transformation that is required to eliminate the elements below the $k$-th pivot can be performed by multiplication with $\mathbf{T}_k$.

$$
\mathbf{T}_k\Amat_{k - 1} = (\mathbf{I} - \mathbf{c}_k\mathbf{e}_k^\top)\Amat_{k - 1} =
\Amat_{k - 1} - \mathbf{c}_k\mathbf{e}_k^\top\Amat_{k - 1}
$$

\lz

We obtain the decomposition by 

$$
\mathbf{U} = \mathbf{T}_n\cdot \mathbf{T}_{n - 1} \cdot ... \cdot \mathbf{T}_1 \cdot \Amat
$$

and

$$
\mathbf{L} = \mathbf{T}_1^{-1} \cdot \mathbf{T}_2^{-1} \cdot ... \cdot \mathbf{T}_n^{-1}
$$

\framebreak

\textbf{Example}:
$$ \scriptsize
n = 4, \quad \mathbf{Ax} = \mathbf{b}, \quad \Amat = \mat{
a_{11} & \cdots & a_{14} \\
\vdots & \ddots & \vdots \\
a_{41} & \cdots & a_{44} }
= \Amat_0
$$

\footnotesize{
\textbf{Note:} Multiplying by $\mathbf{T}_i$ changes the entries of the $(n - i)$ lower right block. To keep the notation readable we write $a_{ij}$ even if the entry was modified by the multiplication. The entries that change in the respective step are highlighted in color. }

\lz

\normalsize
Step 1: \scriptsize
$$
\mathbf{T}_1 \Amat_0 = \mat{
1                & 0 & 0 & 0 \\
-a_{21} / a_{11} & 1 & 0 & 0 \\
-a_{31} / a_{11} & 0 & 1 & 0 \\
-a_{41} / a_{11} & 0 & 0 & 1 }
\Amat_0 =
\mat{
a_{11} & a_{12} & a_{13} & a_{14} \\
0      & \textcolor{red}{a_{22}} & \textcolor{red}{a_{23}} & \textcolor{red}{a_{24}} \\
0      & \textcolor{red}{a_{32}} & \textcolor{red}{a_{33}} & \textcolor{red}{a_{34}} \\
0      & \textcolor{red}{a_{42}} & \textcolor{red}{a_{43}} & \textcolor{red}{a_{44}} }
= \Amat_1
$$

\normalsize

\framebreak

Step 2: \scriptsize
$$
\mathbf{T}_2 \Amat_1 = \mat{
1 & 0                & 0 & 0 \\
0 & 1                & 0 & 0 \\
0 & -\textcolor{red}{a_{32}} / \textcolor{red}{a_{22}} & 1 & 0 \\
0 & -\textcolor{red}{a_{42}} / \textcolor{red}{a_{22}} & 0 & 1 }
\Amat_1 =
\mat{
a_{11} & a_{12} & a_{13} & a_{14} \\
0      & \textcolor{red}{a_{22}} & \textcolor{red}{a_{23}} & \textcolor{red}{a_{24}} \\
0      & 0      & \textcolor{blue}{a_{33}} & \textcolor{blue}{a_{34}} \\
0      & 0      & \textcolor{blue}{a_{43}} & \textcolor{blue}{a_{44}} }
= \Amat_2
$$ \normalsize
Step 3: \scriptsize
$$
\mathbf{T}_3 \Amat_2 = \mat{
1 & 0 & 0                & 0 \\
0 & 1 & 0                & 0 \\
0 & 0 & 1                & 0 \\
0 & 0 & -\textcolor{blue}{a_{43}} / \textcolor{blue}{a_{33}} & 1 }
\Amat_2 =
\mat{
a_{11} & a_{12} & a_{13} & a_{14} \\
0      & \textcolor{red}{a_{22}} & \textcolor{red}{a_{23}} & \textcolor{red}{a_{24}} \\
0      & 0      & \textcolor{blue}{a_{33}} & \textcolor{blue}{a_{34}} \\
0      & 0      & 0      & \textcolor{green}{a_{44}} } = \mathbf{U}
$$

\normalsize

\framebreak

\textbf{Effort} in step $k$ (in multiplications):

\begin{itemize}
\item $(n - k)^2$ multiplications for calculation of $\mathbf{T}_k\Amat_{k - 1}$
\item $(n-k)$ multiplications for $\mathbf{T}_k^{-1}\cdot \underbrace{\mathbf{T}_{k-1}^{-1} \cdot ... \cdot \mathbf{T}_{1}^{-1}}_{\text{already calculated}}$
\end{itemize}

The total effort is therefore

\vspace*{-0.3cm}

\begin{eqnarray*}
& &\sum_{k=1}^n (n - k)^2 + (n - k) = \sum_{k=1}^n n^2 - 2 n k + k^2 + n - k \\
&=& n^3 - 2n \frac{(n + 1)n}{2} + \frac{n(n + 1)(2n + 1)}{6} + n^2 - \frac{n(n+1)}{2} \\
&=& n \cdot \biggl(n^2 - n^2 - n + \frac{1}{3}n^2 + \frac{1}{2} n + \frac{1}{6} + n - \frac{1}{2} n - \frac{1}{2}\biggr) \\
&\approx& \frac{1}{3}n^3 + \order(n).
\end{eqnarray*}

\framebreak

\textbf{Problem}: This only works if all $a_{kk} \not= 0$!

\medskip
\textbf{Pivotization}: $\mathbf{PA} = \mathbf{LU}$\\
\medskip
$\mathbf{P}$ is a permutation matrix which contains the required line switching transformations of the algorithm. \\
\medskip
Switching lines to obtain a more stable algorithm.\\
\medskip
Example: \scriptsize
$$
\Amat_1 = \begin{pmatrix*}[r]
1 &  2 & -1   & 0 \\
0 &  \color{red}{0} & 1/2  & 1 \\
0 &  2 & -1/2 & 3/2 \\
0 & \color{red}{-3} & 5/2  & 0 \end{pmatrix*}
\quad
\mathbf{P}_2 =
\mat{
1 & 0 & 0 & 0 \\
0 & 0 & 0 & \color{red}{1} \\
0 & 0 & 1 & 0 \\
0 & \color{red}{1} & 0 & 0 }
$$

$$
\mathbf{P}_2 \Amat_1 =
\begin{pmatrix*}[r]
1 &  2 & -1   & 0 \\
0 & -3 & 5/2  & 0 \\
0 &  2 & -1/2 & 3/2 \\
0 &  0 & 1/2  & 1 \end{pmatrix*},
$$

\normalsize

then $\mathbf{T}_2\mathbf{P}_2\Amat_1$ etc.

\framebreak

Calculation in general
$$
\Amat_k = \mathbf{T}_k\mathbf{P}_k\Amat_{k - 1}
$$
\medskip
It can be shown
$$
\mathbf{T}_{k - 1}\mathbf{P}_{k - 1} \cdot \ldots \cdot \mathbf{T}_{1}\mathbf{P}_{1} =
  \underbrace{\mathbf{T}_{k - 1} \cdot \ldots \cdot \mathbf{T}_{1}}_{\mathbf{T}} \cdot
  \underbrace{\mathbf{P}_{k - 1} \cdot \ldots \cdot \mathbf{P}_{1}}_{\mathbf{P}}
$$
and thus
$$
\mathbf{TPA} = \mathbf{U} \quad \text{ and } \quad \mathbf{T}^{-1} = \mathbf{L}.
$$

\textbf{Note}: When solving the linear system $\Amat \xv = \mathbf{b}$ the vector $\mathbf{b}$ must also be permuted by $\mathbf{P}$. 

\framebreak


\item Solve  $\mathbf{Ly} = \mathbf{L(Ux)} = \mathbf{Ax} = \mathbf{b}$
\footnotesize
$$
\mat{
1       & 0     & 0       & \cdots & 0 \\
l_{21}  & 1     & 0       & \cdots & 0 \\
l_{31}  & l_{32}& 1       & \cdots & 0 \\
\vdots  & \vdots& \vdots  & \ddots & \vdots \\
l_{n1}  & l_{n2}& l_{n3}  & \cdots & 1 }
\mat{
y_1 \\ y_2 \\ y_3 \\ \vdots \\ y_n
\end{pmatrix} = \begin{pmatrix}
b_1 \\ b_2 \\ b_3 \\ \vdots \\ b_n }
$$
\normalsize

by using \textbf{forward substitution}
$$
y_1 = b_1 \quad \text{ and } \quad y_k = b_k - \sum_{i = 1}^{k - 1}l_{ki}y_i \quad
  \text{ for } \quad k = 2, \ldots, n.
$$

\vspace*{-0.4cm}
\medskip
for our example the result is
\footnotesize
$$
\mat{
1 & 0 & 0 \\
2 & 1 & 0 \\
-\frac{1}{2} & -\frac{1}{2} & 1 }
\mat{
y_1 \\ y_2 \\ y_3
\end{pmatrix} = \begin{pmatrix}
32 \\ 16 \\ 52  } \Rightarrow
\mathbf{y} =
\mat{
32 \\
-48\\
44 }
$$
\normalsize
\framebreak

\item Solve  $\mathbf{Ux} = \mathbf{y}$\\
\medskip
Since $\mathbf{y}$ is now known from step 2\\
\footnotesize
$$
\mat{
u_{11}  & \cdots  & \cdots & u_{1n} \\
0       & \ddots  & \ddots & \vdots \\
\vdots  & \ddots  & \ddots  & \vdots \\
0       & \cdots  & 0       & u_{nn} }
\mat{
  x_1 \\ x_2 \\ x_3 \\ \vdots \\ x_n
  \end{pmatrix} = \begin{pmatrix}
  y_1 \\ y_2 \\ y_3 \\ \vdots \\ y_n }
$$
\normalsize

we can calculate $\mathbf{x}$ using \textbf{back substitution}:
\footnotesize
$$
x_i = \frac{1}{u_{ii}} \left( y_i - \sum_{k = i + 1}^n u_{ik}x_k \right) \quad
  \text{ for } \quad i = n - 1, n - 2, \ldots, 1.
$$
\normalsize
For our example the solution to the LES is:
\footnotesize
$$
\mat{
2 & 8 & 1 \\
0 & -12 & -3 \\
0 & 0 & 11 }
\mat{
x_1 \\ x_2 \\ x_3
\end{pmatrix} = \begin{pmatrix}
32 \\ -48 \\ 44  } \Rightarrow
\mathbf{x} =
\mat{
2 \\
3\\
4 }
$$
\normalsize
\end{enumerate}
\framebreak

\textbf{Effort} of \textbf{forward substitution}:\\

\medskip

In step $k$,  $k - 1$ multiplications are performed. If only multiplications are taken into consideration, the resulting effort is

\begin{eqnarray*}
\sum_{k=2}^{n} (k - 1) = \sum_{k=1}^{n - 1} k = \frac{(n- 1)n}{2} = \frac{1}{2}n^2 - \frac{n}{2}
\end{eqnarray*}

\lz

\textbf{Effort} of \textbf{back substitution}:\\
\medskip
Similar to forward substitution, the required effort is

\begin{eqnarray*}
\frac{1}{2}n^2 + \frac{n}{2}
\end{eqnarray*}

\framebreak


\end{vbframe}

\begin{vbframe} {Properties LU decomposition}

\begin{itemize}
\item \enquote{Interpretation} of the Gaussian elimination as matrix decomposition
% \item $\Amat$ is split into a lower triangle matrix $\mathbf{L}$ and a lower triangle matrix $\mathbf{U}$.
\item Numerically stable during pivoting
\item \textbf{Existence:} For each \textbf{regular} matrix $\Amat$ there is a permutation matrix $\mathbf{P}$, a normalized lower triangle matrix $\mathbf{L} \in \R^{n \times n}$ and a normalized upper triangular matrix $\mathbf{U} \in \R^{n \times n}$ such that 
$$
\mathbf{P} \cdot \Amat = \mathbf{L} \cdot \mathbf{U}
$$
\item Runtime behavior:
\begin{itemize}
\item Decomposition of the matrix: $\frac{n^3}{3} + \order(n)$ multiplications. 
\item Forward and back substitution: $n^2$
\end{itemize}
% $\to$ Gesamtaufwand von $\order(n^3)$
% \item Rechnung für Zerlegung kann auf Speicher der Matrix $\Amat$ durchgeführt werden \\
% $\to$ kein zusätzlicher Speicher nötig
\end{itemize}





\end{vbframe}


\endlecture
\end{document}







