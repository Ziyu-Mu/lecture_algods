
%<<setup-child, include = FALSE>>=
%library(knitr)
%options(digits = 16)

%library(RCurl)
%library(XML)
%library(tm)
%library(NMF)
%library(microbenchmark)
%library(ggplot2)
%library(wordcloud)
%set_parent("../style/preamble.Rnw")
%@


\newcommand{\xdownarrow}[1]{%
  {\left\downarrow\vbox to #1{}\right.\kern-\nulldelimiterspace}
}

\newcommand{\grey}[1]{\textcolor{grey}{#1}}
\newcommand{\red}[1]{\textcolor{red}{#1}}

\input{../../2021/style/preamble4tex}
% dependencies: amsmath, amssymb, dsfont
% math spaces
\ifdefined\N
\renewcommand{\N}{\mathds{N}} % N, naturals
\else \newcommand{\N}{\mathds{N}} \fi
\newcommand{\Z}{\mathds{Z}} % Z, integers
\newcommand{\Q}{\mathds{Q}} % Q, rationals
\newcommand{\R}{\mathds{R}} % R, reals
\ifdefined\C
\renewcommand{\C}{\mathds{C}} % C, complex
\else \newcommand{\C}{\mathds{C}} \fi
\newcommand{\continuous}{\mathcal{C}} % C, space of continuous functions
\newcommand{\M}{\mathcal{M}} % machine numbers
\newcommand{\epsm}{\epsilon_m} % maximum error

% counting / finite sets
\newcommand{\setzo}{\{0, 1\}} % set 0, 1
\newcommand{\setmp}{\{-1, +1\}} % set -1, 1
\newcommand{\unitint}{[0, 1]} % unit interval

% basic math stuff
\newcommand{\xt}{\tilde x} % x tilde
\newcommand{\argmin}{\mathop{\mathrm{arg\,min}}} % argmin
\newcommand{\argmax}{\mathop{\mathrm{arg\,max}}} % argmax
\newcommand{\argminlim}{\argmin\limits} % argmin with limits
\newcommand{\argmaxlim}{\argmax\limits} % argmax with limits
\newcommand{\sign}{\operatorname{sign}} % sign, signum
\newcommand{\I}{\mathbb{I}} % I, indicator
\newcommand{\order}{\mathcal{O}} % O, order
\newcommand{\bigO}{\mathcal{O}} % Big-O Landau
\newcommand{\littleo}{{o}} % Little-o Landau
\newcommand{\pd}[2]{\frac{\partial{#1}}{\partial #2}} % partial derivative
\newcommand{\floorlr}[1]{\left\lfloor #1 \right\rfloor} % floor
\newcommand{\ceillr}[1]{\left\lceil #1 \right\rceil} % ceiling
\newcommand{\indep}{\perp \!\!\! \perp} % independence symbol

% sums and products
\newcommand{\sumin}{\sum\limits_{i=1}^n} % summation from i=1 to n
\newcommand{\sumim}{\sum\limits_{i=1}^m} % summation from i=1 to m
\newcommand{\sumjn}{\sum\limits_{j=1}^n} % summation from j=1 to p
\newcommand{\sumjp}{\sum\limits_{j=1}^p} % summation from j=1 to p
\newcommand{\sumik}{\sum\limits_{i=1}^k} % summation from i=1 to k
\newcommand{\sumkg}{\sum\limits_{k=1}^g} % summation from k=1 to g
\newcommand{\sumjg}{\sum\limits_{j=1}^g} % summation from j=1 to g
\newcommand{\summM}{\sum\limits_{m=1}^M} % summation from m=1 to M
\newcommand{\meanin}{\frac{1}{n} \sum\limits_{i=1}^n} % mean from i=1 to n
\newcommand{\meanim}{\frac{1}{m} \sum\limits_{i=1}^m} % mean from i=1 to n
\newcommand{\meankg}{\frac{1}{g} \sum\limits_{k=1}^g} % mean from k=1 to g
\newcommand{\meanmM}{\frac{1}{M} \sum\limits_{m=1}^M} % mean from m=1 to M
\newcommand{\prodin}{\prod\limits_{i=1}^n} % product from i=1 to n
\newcommand{\prodkg}{\prod\limits_{k=1}^g} % product from k=1 to g
\newcommand{\prodjp}{\prod\limits_{j=1}^p} % product from j=1 to p

% linear algebra
\newcommand{\one}{\bm{1}} % 1, unitvector
\newcommand{\zero}{\mathbf{0}} % 0-vector
\newcommand{\id}{\bm{I}} % I, identity
\newcommand{\diag}{\operatorname{diag}} % diag, diagonal
\newcommand{\trace}{\operatorname{tr}} % tr, trace
\newcommand{\spn}{\operatorname{span}} % span
\newcommand{\scp}[2]{\left\langle #1, #2 \right\rangle} % <.,.>, scalarproduct
\newcommand{\mat}[1]{\begin{pmatrix} #1 \end{pmatrix}} % short pmatrix command
\newcommand{\Amat}{\mathbf{A}} % matrix A
\newcommand{\Deltab}{\mathbf{\Delta}} % error term for vectors

% basic probability + stats
\renewcommand{\P}{\mathds{P}} % P, probability
\newcommand{\E}{\mathds{E}} % E, expectation
\newcommand{\var}{\mathsf{Var}} % Var, variance
\newcommand{\cov}{\mathsf{Cov}} % Cov, covariance
\newcommand{\corr}{\mathsf{Corr}} % Corr, correlation
\newcommand{\normal}{\mathcal{N}} % N of the normal distribution
\newcommand{\iid}{\overset{i.i.d}{\sim}} % dist with i.i.d superscript
\newcommand{\distas}[1]{\overset{#1}{\sim}} % ... is distributed as ...


\begin{document}

\lecturechapter{7}{Cholesky Decomposition}
\lecture{CIM1 Statistical Computing}




\begin{vbframe}{Cholesky decomposition}

\textbf{Aim:} Solve LES of the form $\Amat \xv = \boldsymbol{b}$\\
\medskip
with $\Amat \in \R^{n \times n}$, $\Amat$ positive-definite
\medskip
\begin{enumerate}
\item Write $\Amat$ as $\Amat = \mathbf{L}\mathbf{L}^\top$\\
\item Solve $\mathbf{Ly} = \mathbf{b}$ by forward substitution
\item Solve $\mathbf{L}^\top \mathbf{x}=\mathbf{y}$ by back substitution
\framebreak
\end{enumerate}

Example:
Let $\Amat \xv = \boldsymbol{b}$ be a LES

$$
\begin{pmatrix*}[r]
4 & 2 & 2 & 2\\
2 & 5 & 3 & 3 \\
2 & 3 & 11&5\\
2 & 3 & 5 & 19\end{pmatrix*}
\mat{
x_1 \\ x_2 \\ x_3\\ x_4
\end{pmatrix} = \begin{pmatrix}
22 \\ 33 \\ 61 \\ 99 }
$$

\end{vbframe}

\begin{frame}{Cholesky decomposition}
\begin{enumerate}
\item Write $\Amat$ as $\Amat = \mathbf{L}\mathbf{L}^\top$
\scriptsize


 \only<1>{
 \begin{eqnarray*}
\mat{
\textcolor{blue}{l_{11}} & \red{0}  & \red{0}  & \red{0}  \\
l_{21} & l_{22} & 0 & 0 \\
l_{31} & l_{32} & l_{33} & 0 \\
l_{41} & l_{42} & l_{43} & l_{44} }
\mat{
\textcolor{blue}{l_{11}} & l_{21} & l_{31} & l_{41} \\
\red{0} & l_{22} & l_{32} & l_{42} \\
\red{0} & 0 & l_{33} & l_{43} \\
\red{0} & 0 & 0 & l_{44} }
= \mat{
\textcolor{blue}{4} & 2 & 2 & 2 \\
2 & 5 & 3 & 3 \\
2 & 3 & 11 & 5 \\
2 & 3 & 5 & 19 }
\end{eqnarray*}
\vspace*{-0.4cm}
\begin{eqnarray*}
\textcolor{blue}{l_{11}^2 = a_{11}} &\rightarrow& \textcolor{blue}{l_{11} = \sqrt{a_{11}} = \sqrt{4} = 2}
\end{eqnarray*}

}

\only<2>{
 \begin{eqnarray*}
\mat{
l_{11} & 0 & 0 & 0 \\
\textcolor{blue}{l_{21}} & \textcolor{blue}{l_{22}} & \red{0} & \red{0} \\
l_{31} & l_{32} & l_{33} & 0 \\
l_{41} & l_{42} & l_{43} & l_{44} }
\mat{
\textcolor{blue}{l_{11}} & l_{21} & l_{31} & l_{41} \\
\red{0} & l_{22} & l_{32} & l_{42} \\
\red{0} & 0 & l_{33} & l_{43} \\
\red{0} & 0 & 0 & l_{44} }
= \mat{
4 & 2 & 2 & 2 \\
\textcolor{blue}{2} & 5 & 3 & 3 \\
2 & 3 & 11 & 5 \\
2 & 3 & 5 & 19 }
\end{eqnarray*}

\vspace*{-0.4cm}
\begin{eqnarray*}
l_{11}^2 = a_{11} &\rightarrow& l_{11} = \sqrt{a_{11}} = \sqrt{4} = 2 \\
\textcolor{blue}{l_{21} \cdot l_{11} = a_{21}} &\rightarrow& \textcolor{blue}{l_{21} = \frac{a_{21}}{l_{11}}  = \frac{2}{2} = 2}
\end{eqnarray*}
}

 \only<3>{
 \begin{eqnarray*}
\mat{
l_{11} & 0 & 0 & 0 \\
\textcolor{blue}{l_{21}} & \textcolor{blue}{l_{22}} & \red{0} & \red{0} \\
l_{31} & l_{32} & l_{33} & 0 \\
l_{41} & l_{42} & l_{43} & l_{44} }
\mat{
l_{11} & \textcolor{blue}{l_{21}} & l_{31} & l_{41} \\
0 & \textcolor{blue}{l_{22}} & l_{32} & l_{42} \\
0 & \red{0} & l_{33} & l_{43} \\
0 & \red{0} & 0 & l_{44} }
= \mat{
4 & 2 & 2 & 2 \\
2 & \red{5} & 3 & 3 \\
2 & 3 & 11 & 5 \\
2 & 3 & 5 & 19 }
\end{eqnarray*}

\vspace*{-0.4cm}
\begin{eqnarray*}
l_{11}^2 = a_{11} &\rightarrow& l_{11} = \sqrt{a_{11}} = \sqrt{4} = 2 \\
l_{21} \cdot l_{11} = a_{21} &\rightarrow& l_{21} = \frac{a_{21}}{l_{11}}  = \frac{2}{2} = 2 \\
\textcolor{blue}{l_{22}^2 + l_{21}^2 = a_{22}} &\rightarrow& \textcolor{blue}{l_{22} = \sqrt{a_{22} - l_{21}^2} = \sqrt{5 - 1^2} = 2} \\
\end{eqnarray*}
}

 \only<4>{
 \begin{eqnarray*}
\mat{
l_{11} & 0 & 0 & 0 \\
l_{21} & l_{22} & 0 & 0 \\
\textcolor{blue}{l_{31}} & \textcolor{blue}{l_{32}} & \textcolor{blue}{l_{33}} & \red{0} \\
l_{41} & l_{42} & l_{43} & l_{44} }
\mat{
\textcolor{blue}{l_{11}} & l_{21} & l_{31} & l_{41} \\
\red{0} & l_{22} & l_{32} & l_{42} \\
\red{0} & 0 & l_{33} & l_{43} \\
\red{0} & 0 & 0 & l_{44} }
= \mat{
4 & 2 & 2 & 2 \\
2 & 5 & 3 & 3 \\
\textcolor{blue}{2} & 3 & 11 & 5 \\
2 & 3 & 5 & 19 }
\end{eqnarray*}

\vspace*{-0.4cm}
\begin{eqnarray*}
l_{11}^2 = a_{11} &\rightarrow& l_{11} = \sqrt{a_{11}} = \sqrt{4} = 2 \\
l_{21} \cdot l_{11} = a_{21} &\rightarrow& l_{21} = \frac{a_{21}}{l_{11}}  = \frac{2}{2} = 2 \\
l_{22}^2 + l_{21}^2 = a_{22} &\rightarrow& l_{22} = \sqrt{a_{22} - l_{21}^2} = \sqrt{5 - 1^2} = 2 \\
\textcolor{blue}{l_{31} \cdot l_{11} = a_{31}}  &\rightarrow& \textcolor{blue}{l_{31} = \frac{a_{31}}{l_{11}} = \frac{2}{2} = 1 } \\
&\vdots &
\end{eqnarray*}
General formula: $ l_{jj} = \left(a_{jj} - \sum_{k = 1}^{j - 1}l_{jk}^2\right)^{\frac{1}{2}} \quad
l_{ij} = \frac{1}{l_{jj}}\left(a_{ij} - \sum_{k = 1}^{j - 1}l_{ik}l_{jk}\right)$
}

\normalsize
\end{enumerate}

\end{frame}

\begin{vbframe}{Cholesky decomposition}
\begin{enumerate}
\setcounter{enumi}{1}

\item Solve $\mathbf{Ly} = \mathbf{b}$ by forward substitution


% [4ex]
% \mathbf{L}\mathbf{L}^{\prime}\mathbf{x} = \mathbf{y} &\Leftrightarrow& \begin{cases}
% \mathbf{L}\mathbf{z} = \mathbf{y} & (1) \\
% \mathbf{L}^{\prime}\mathbf{x} = \mathbf{z} & (2)
% \end{cases} \\[4ex]
\begin{eqnarray*}
\mat{
  2 & 0 & 0 & 0 \\
  1 & 2 & 0 & 0 \\
  1 & 1 & 3 & 0 \\
  1 & 1 & 1 & 4 }
\mat{
  y_1 \\
  y_2 \\
  y_3 \\
  y_4 }
&=& \mat{
  22 \\
  33 \\
  61 \\
  99 } \\
\end{eqnarray*}

\vspace*{-0.5cm}

\begin{eqnarray*}
\mat{
  2y_1 \\
  y_1 + 2y_2 \\
  y_1 + y_2 + 3y_3 \\
  y_1 + y_2 + y_3 + 4y_4 }
&=& \mat{
  22 \\
  33 \\
  61 \\
  99 } \\
\end{eqnarray*}
\begin{eqnarray*}
\Rightarrow&  y_1 = 11, y_2 = 11, y_3 = 13, y_4 = 16
\end{eqnarray*}

\normalsize
\framebreak
\medskip
\item Solve $\mathbf{L}^\top \mathbf{x}=\mathbf{y}$ by back substitution
\begin{eqnarray*}
\mat{
2 & 1 & 1 & 1 \\
0 & 2 & 1 & 1 \\
0 & 0 & 3 & 1 \\
0 & 0 & 0 & 4 }
\mat{
x_1 \\
x_2 \\
x_3 \\
x_4 }
&=& \mat{
11 \\
11 \\
13 \\
16 } \\
\end{eqnarray*}
\begin{eqnarray*}
&\Rightarrow& x_4 = 4, x_3 = 3, x_2 = 2, x_1 = 1
\end{eqnarray*}
\end{enumerate}
\framebreak

Calculation of the lower triangular matrix ($\mathbf{L}$):
\begin{eqnarray*}
\mat{
l_{11} & 0 & 0 & 0 \\
l_{21} & l_{22} & 0 & 0 \\
l_{31} & l_{32} & l_{33} & 0 \\
l_{41} & l_{42} & l_{43} & l_{44} }
\mat{
l_{11} & l_{21} & l_{31} & l_{41} \\
0 & l_{22} & l_{32} & l_{42} \\
0 & 0 & l_{33} & l_{43} \\
0 & 0 & 0 & l_{44} }
= \mat{
a_{11} & a_{12} & a_{13} & a_{14} \\
a_{21} & a_{22} & a_{23} & a_{24} \\
a_{31} & a_{32} & a_{33} & a_{34} \\
a_{41} & a_{42} & a_{43} & a_{44} }
\end{eqnarray*}
Thus the entries of $\mathbf{L}$ (j rows, i columns) result from


$$
l_{ij} =
\begin{cases}
0 & \text{for } i < j \\[5pt]
(a_{jj} - \sum_{k = 1}^{j - 1}l_{jk}^2)^{\frac{1}{2}} & \text{for } i = j \\[5pt]
\frac{1}{l_{jj}} (a_{ij} - \sum_{k = 1}^{j - 1}l_{ik}l_{jk}) & \text{for } i > j
\end{cases}
$$
\medskip
\textbf{Important: } Order of calculation (row by row) matters!\\
$\rightarrow$ $l_{11}$, $l_{21}$, $l_{22}$, $l_{31}$, $l_{32}$, $l_{33}$,..., $l_{nn}$


% \begin{eqnarray*}
% l_{jj} = \left(a_{jj} - \sum_{k = 1}^{j - 1}l_{jk}^2\right)^{\frac{1}{2}} & & l_{ij} =
%   \frac{1}{l_{jj}}\left(a_{ij} - \sum_{k = 1}^{j - 1}l_{ik}l_{jk}\right).
% \end{eqnarray*}

\framebreak

\begin{algorithm}[H]
  \caption{Cholesky decomposition}
  \begin{algorithmic}[1]
  \For {$j = 1 \text{ to } n$}
    \State $l_{jj} = \left(a_{jj} - \sum_{k = 1}^{j - 1}l_{jk}^2\right)^{\frac{1}{2}}$
    \For {$i=j+1 \text{ to } n$}
      \State $l_{ij} =
  \frac{1}{l_{jj}}\left(a_{ij} - \sum_{k = 1}^{j - 1}l_{ik}l_{jk}\right)$
    \EndFor
  \EndFor
  \end{algorithmic}
\end{algorithm}

If we consider only the (dominant) multiplications, we count in each step of the outer loop

\begin{itemize}
\item For diagonal elements: $(j-1)$ multiplications
\item For non-diagonal elements: $(n-j)(j-1)$ multiplications
\end{itemize}

In total, we estimate the computational effort with

\vspace*{-0.4cm}
\begin{eqnarray*}
 & &\sum_{j=1}^n [(j - 1) + (n - j)(j - 1)] \\
&=& \sum_{j=1}^n [j - 1 + nj - n - j^2 + j] = \sum_{j=1}^n[(n + 2) j - 1 - j^2] \\
&=&
n \frac{(n + 2) (n + 1)}{2} - n - n\frac{(n+1)(2n+ 1)}{6} \\
&=& n \cdot \frac{3(n + 2)(n + 1) - 6 - (n + 1)(2n + 1)}{6} \\
&=& n \cdot \frac{3n^2 + 9n + 6 - 6 - 2n^2 - 2n - n - 1} {6}\\
&\approx& \frac{1}{6}n^3 + \order(n^2) \quad \text{for large }n
\end{eqnarray*}

\framebreak
\end{vbframe}

\begin{vbframe}{Properties of Cholesky decomposition}

\begin{itemize}
\item Most important procedure for positive-definite matrices
\item Algorithm is always stable (no pivoting necessary)
\item \textbf{Existence} and \textbf{uniqueness}: The Cholesky decomposition exists and is unique for a positive-definite matrix $\Amat$
\item Runtime behavior:
\begin{itemize}
\item Decomposition of the matrix: $\frac{n^3}{6} + \order(n^2)$ multiplications
\item Forward and back substitution: $n^2$
\end{itemize}
% $\to$ Insgesamt $\order(n^3)$
\end{itemize}

% plus $n$ mal Berechnung einer Wurzel.
% %, und ist \textbf{immer stabil}.


%>\end{vbframe}

%\begin{vbframe}{Beispiel Cholesky-Zerlegung}

<<include = FALSE>>=
cholesky = function(a) {
  n = nrow(a)
  l = matrix(0, nrow = n, ncol = n)
  for (j in 1:n) {
    l[j, j] = (a[j, j] - sum(l[j, 1:(j - 1)]^2))^0.5
    if (j < n) {
      for (i in (j + 1):n) {
        l[i, j] = (a[i, j] -
          sum(l[i, 1:(j - 1)] * l[j, 1:(j - 1)])) / l[j, j]
      }
    }
  }
  return(l)
}
@

%\framebreak

<<include = FALSE>>=
A = crossprod(matrix(runif(16), 4, 4))
cholesky(A)
@

%\framebreak

<<include = FALSE>>=
t(chol(A))
@

%\framebreak

<<include = FALSE>>=
A = crossprod(matrix(runif(1e+06), 1e+03, 1e+03))
system.time(cholesky(A))
system.time(chol(A))
@

%\begin{Schunk}
% \begin{Sinput}
% R> x = crossprod(matrix(runif(1e+06), 1e+03, 1e+03))
% R> system.time(cholesky(x))
% \end{Sinput}
% \begin{Soutput}
%    user  system elapsed
%   6.016   0.004   6.027
% \end{Soutput}
% \begin{Sinput}
% R> system.time(chol(x))
% \end{Sinput}
% \begin{Soutput}
%    user  system elapsed
%   0.160   0.000   0.159
% \end{Soutput}
%\end{Schunk}

\framebreak

\end{vbframe}
\begin{vbframe}{Application ex.: Multivariate Gaussian}
\textbf{Target:} Efficient evaluation of the density of a normal distribution.
\medskip

The density of the $d$-dimensional multivariate normal distribution is
\medskip
$$
f(\mathbf{x})= \frac{1}{(2\pi)^\frac{d}{2}|\boldsymbol{\Sigma}|^\frac{1}{2}} \exp\{-\frac{1}{2}(\mathbf{x}-\boldsymbol{\mu})^\top \boldsymbol{\Sigma}^{-1} (\mathbf{x}-\boldsymbol{\mu})\}
$$
\medskip
with $\mathbf{x}\in\R^{d}$, $\text{Cov}(\mathbf{x}) = \boldsymbol{\Sigma}, \boldsymbol{\Sigma}$ \textbf{positive-definite}.\\
\medskip
With $\mathbf{z} = \mathbf{x}-\boldsymbol{\mu}, \mathbf{z}\in\R^{d} $ we obtain:
$$
(\mathbf{x}-\boldsymbol{\mu})^\top \boldsymbol{\Sigma^{-1}} (\mathbf{x}-\boldsymbol{\mu}) = \mathbf{z}^\top \boldsymbol{\Sigma}^{-1}\mathbf{z}
$$

\lz

\textbf{Problem:} Calculation of $\boldsymbol{\Sigma^{-1}}$ is numerically unstable and requires a long time.
\framebreak

\textbf{Solution:} Use Cholesky decomposition to avoid inverting $\boldsymbol{\Sigma}^{-1}$.
%\medskip
Write $\boldsymbol{\Sigma}$ as  $\boldsymbol{\Sigma} = \mathbf{L}\mathbf{L}^\top$, $rank(\mathbf{L}) = d$.\\
\medskip
Thus it holds:
\begin{eqnarray*}
  \mathbf{z}^\top \boldsymbol{\Sigma}^{-1} \mathbf{z}  &=&  \mathbf{z}^\top (\mathbf{L}\mathbf{L}^\top)^{-1}\mathbf{z} \\
                                                      &=&  \mathbf{z}^\top (\mathbf{L}^\top)^{-1} \mathbf{L}^{-1} \mathbf{z} \\
                                                      &=&  (\mathbf{L}^{-1} \mathbf{z})^\top\mathbf{L}^{-1} \mathbf{z}\\
                                                      &=&  \mathbf{v}^\top\mathbf{v}
\end{eqnarray*}
with $\mathbf{v} = \mathbf{L}^{-1} \mathbf{z}$, $\mathbf{v}\in\R^{d}.$\\
\medskip
%\framebreak
To avoid inverting $\mathbf{L}$ we can calculate $\mathbf{v}$ as a solution of the LES

\vspace*{-0.2cm}
$$
\mathbf{L}\mathbf{v} = \mathbf{z}
$$


\medskip
Then we can calculate $\mathbf{v}^{T}\mathbf{v}$ as a scalar product of two $d$-dimensional vectors.
\end{vbframe}



\endlecture
\end{document}







